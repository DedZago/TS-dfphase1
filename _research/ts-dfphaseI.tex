\documentclass{article}
\usepackage{global-layout-notes}
\usepackage{global-statmacros}
\bibliography{~/Documents/MEGA/main/courses/biblio.bib}
\author{Daniele Zago$^1$}
\title{Time series dfphase~I}
\date{%
\smaller
$^1$ \textit{Department of Statistics, University of Padua, Padua, Italy}\\%
v0.1 on \today}
\begin{document}
\maketitle

\paragraph{Summary.}
To allow for nonparametric testing of the stability of correlated data, the \texttt{mphase1} procedure needs to be modified.
 This involves substituting the permutation hypothesis test with a hypothesis test based on block bootstrap, and removing the autoregressive effect from the mean estimate.
 The proposed procedure follows these steps: firstly, the block length $b$ is selected based on the maximum of median block sizes (if $ n > 1$) among each marginal process.
 The data is then transformed nonparametrically using the signed rank transformation.
 At each iteration $k$, step and isolated shifts are added to the model and the most promising one is included in the regression.
 Variable selection is performed by including a maximum of $q$ lagged values of the variable and using the LARS algorithm to select the first dummy regressor that is included among the candidate ones.
 A full vector autoregressive model is avoided due to the large number of parameters it entails.
 Once the regressor is added to the design matrix, the model statistic is computed by calculating the explained variance with both autoregressive and selected dummy variables.
 It is not necessary for the statistics to have exactly zero mean and unitary variance; they only need to be approximately scaled to unit \citep{pesarin2016}.

One way to accelerate computation is to apply the technique proposed by \citet{ding2020} in the following way:
\begin{enumerate}[label=\arabic*)]
    \item Compute $ J = 50$ values of the statistics $ T_{k}$ and calculate preliminary values $ \E[T_{k}]$ and $ \V[T_{k}]$. 
    \item Then, calculate preliminary $ p$-value using the stopping rule from \citet{ding2020} with small $ \varepsilon$, e.g. 0.005.
    \item If the method did not accept/reject the hypothesis, then do:
        \begin{enumerate}
            \item Calculate new statistics $ T_{k}, k = 1, \ldots, K$.
            % \item Update $ \E[T_{k}]$ and $ \V[T_{k}]$ for each $ k$ using the \href{https://math.stackexchange.com/questions/374881/recursive-formula-for-variance}{recursive formulas}.
            \item Update $ p$-value for the stopping rule in 2).
        \end{enumerate}
\end{enumerate}


\printbibliography
\end{document}

