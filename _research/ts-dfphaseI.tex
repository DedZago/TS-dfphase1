\documentclass{article}
\usepackage{global-layout-notes}
\usepackage{global-statmacros}
\bibliography{~/Documents/MEGA/main/biblio}
\author{Daniele Zago$^1$}
\title{Time series dfphase~I}
\date{%
\smaller
$^1$ \textit{Department of Statistics, University of Padua, Padua, Italy}\\%
v0.1 on \today}
\begin{document}
\maketitle

\paragraph{Summary.} Modify the \texttt{mphase1} procedure to allow for nonparametric testing of the stability of correlated data.
The extensions required to do so are the substitution of the permutation hypothesis test with a hypothesis test based on block bootstrap; and to remove the autoregressive effect from the mean estimate.
The proposed procedure is as follows: first, select the block length $ b$ so that a block bootstrap can be applied to the data.
This is done by selecting the maximum of the median block sizes (if there are multiple rational subgroups) among each marginal process.
It is better to avoid a vector autoregressive model, because the number of parameters would be too large.
Then, apply the signed rank transformation to the data in order for the procedure to be nonparametric.
Then, at each iteration $ k$, the step and isolated shifts are added to the model and the most promising one is included in the regression.
Variable selection might be performed by using the LARS algorithm and selecting the best dummy regressor among the candidates.
This simply amounts to finding the potential regressor with highest correlation with the response variable.
Then, the regressor is added to the design matrix and standard linear regression is used to estimate the predicted value.


\printbibliography
\end{document}

